%%=============================================================================
%% Inleiding
%%=============================================================================

\chapter{\IfLanguageName{dutch}{Inleiding}{Introduction}}%
\label{ch:inleiding}

Binnen Delaware is twee jaar geleden software ontwikkeld om betalende functionaliteiten binnen Azure niet meer te hoeven gebruiken. Deze software heeft echter dooheen de tijd extra functionaliteiten toegewezen gekregen en kan nu ook data opvragen en bewerken binnen een databank die zich in Azure bevind.

Nu wordt er binnen Delaware bekeken wat mogelijks volgende bruikbare uitbreidingen zijn en werden de ogen gericht op GraphQL. Dit zou mogelijks kunnen samenwerken met de Data-Accelerator en andere resources binnen de omgeving van een accelerator. Het doel is hierbij om te onderzoeken in hoeverre GraphQL kan geïmplementeerd worden.

Dit leverde de volgende onderzoeksvraag op: "Hoe kan GraphQL geïmplementeerd worden in een zelfontwikkelde accelerator?" Dit wordt odnersteun door volgende doelstellingen:

\begin{itemize}
  \item Documenteren van de Data-Accelerator.
  \item Omvormen van benodigde resources tot BICEP templates.
  \item Opstelllen van een pipeline om de BICEP templates in een omgeving te realiseren.
  \item Mogelijkheid tot personalisatie toevoegen.
  \item GraphQL toevoegen aan de omgeving op een verstaanbare manier.
\end{itemize}

\section{\IfLanguageName{dutch}{Probleemstelling}{Problem Statement}}%
\label{sec:probleemstelling}

Delaware beschikt momenteel over een Data-Accelerator die bij verscheidene projecten gebruikt kan worden. Momenteel moet echter de omgeving om deze te kunnen gebruiken volledig manueel aangemaakt en getest worden.  Dit kost echter veel tijd en moet bij elke nieuwe omgeving opnieuw opgesteld worden. Men wil dit ook uitbreiden met GraphQL functionaliteit maar heeft nog geen tijd ter beschikking gehad om dit te onderzoeken of dit in de eerste plek wel mogelijk is.

De doelgroep van dit onderzoek is dan ook Delaware en meer bepaald het Microsoft Integration team. Deze spitst zich toe op Microsoft Azure en maakt al reeds bij enkele projecten gebruik van de accelerator. Doorheen het gebruik hiervan zijn al enkele extra functionaliteiten toegevoegd maar er ontbreken nog uitbreidingen om dit als een starterpakket mee te kunnen geven om generiek te werk te gaan binnen het hele team.

\section{\IfLanguageName{dutch}{Onderzoeksvraag}{Research question}}%
\label{sec:onderzoeksvraag}

Via dit onderzoek wil men bij Delaware teweten komen of GraphQL functionaliteit kan toegevoegd worden aan de Data-Accelerator werkomgeving op een manier dat dit gemakkelijk herbruikbaar is over meerdere projecten.

De hoofdonderzoeksvraag luidt daarom als volgt: "Hoe kan GraphQL geïmplementeerd worden in een zelfontwikkelde accelerator?"

Doorheen het verloop van de bachelorproef zal er zich verdiept worden in GraphQL en de Data-Accelerator met diens werkomgeving. Daarbij worden volgende deelonderzoeksvragen beantwoord.

\begin{itemize}
    \item Wat is GraphQL?
    \item Wat is een Data-Accelerator?
    \item Kan de accelerator GraphQL benutten?
    \item Welke integratiemogelijkheden zijn er voor GraphQL binnen Data-Accelerator?
\end{itemize}

\section{\IfLanguageName{dutch}{Onderzoeksdoelstelling}{Research objective}}%
\label{sec:onderzoeksdoelstelling}

Aan de hand van documentatie de werking van de Data-Accelerator in beeld brengen om dan via een geautomatiseerde methode een bruikbare werkomgeving op te stellen. Deze moet alle benodigdheden bevatten om de Data-Accelerator te kunnen gebruiken en om GraphQL te kunnen implementeren op een tijds-efficiënte wijze. Wanneer deze doelstellingen behaalt is, kan dit onderzoek als geslaagd bezien worden.

\section{\IfLanguageName{dutch}{Opzet van deze bachelorproef}{Structure of this bachelor thesis}}%
\label{sec:opzet-bachelorproef}

% Het is gebruikelijk aan het einde van de inleiding een overzicht te
% geven van de opbouw van de rest van de tekst. Deze sectie bevat al een aanzet
% die je kan aanvullen/aanpassen in functie van je eigen tekst.

De rest van deze bachelorproef is als volgt opgebouwd:

In Hoofdstuk~\ref{ch:stand-van-zaken} wordt een overzicht gegeven van de stand van zaken binnen het onderzoeksdomein, op basis van een literatuurstudie.

In Hoofdstuk~\ref{ch:methodologie} wordt de methodologie toegelicht en worden de gebruikte onderzoekstechnieken besproken om een antwoord te kunnen formuleren op de onderzoeksvragen.

% TODO: Vul hier aan voor je eigen hoofstukken, één of twee zinnen per hoofdstuk

In Hoofdstuk~\ref{ch:conclusie}, tenslotte, wordt de conclusie gegeven en een antwoord geformuleerd op de onderzoeksvragen. Daarbij wordt ook een aanzet gegeven voor toekomstig onderzoek binnen dit domein.