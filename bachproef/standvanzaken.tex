\chapter{\IfLanguageName{dutch}{Stand van zaken}{State of the art}}%
\label{ch:stand-van-zaken}

\section{\IfLanguageName{dutch}{GraphQL}{GraphQL}}%
\label{sec:GraphQL}

In deze sectie van de literatuurstudie spitsen we ons toe op de eerste deelvraag van dit onderzoek: “Wat is GraphQL?”
De hier opvolgende secties zijn opgesteld naarmate de benodigde onderdelen die helpen een beter beeld te scheppen over de werking van GraphQL en hoe deze software tot stand is gekomen.
Dit wordt gevolgd door toelichting van de Data Accelerator. Hierin zal de werking en toepassing van de software binnen een werkomgeving in kaart gebracht worden, zodanig dat met begrip van de GraphQL secties er een beeld gevormd kan worden hoe deze twee programma's samen kunnen werken.

Het volgende deel zal dan de mogelijke voordelen bespreken van GraphQL en deze koppelen aan de Data Accelerator. Hiervoor zal er gekeken worden naar reeds bestaande onderzoeken die gelijkaardige materie duiding geven.
Tot slot zal er binnen de literatuurstudie besproken worden welke mogelijke integratie opties er zijn binnen de Data Accelerator en een synopsis van desbetreffende mogelijkheden.

\subsection{\IfLanguageName{dutch}{Gegevens ophalen}{Fetching data}}%
\label{sec:Gegevens ophalen}
In 2012 creëerde Facebook GraphQL omdat men binnen het bedrijf op zoek was naar een betere manier om data op te halen, om zo hun mobiele applicaties herop te bouwen. Voorheen werd er bij hun iOS en Android apps gebruik gemaakt van een omvorming van hun website. Dit zorgde op termijn dat naarmate de applicaties uitgebreider werden, er zich meer problemen rondom geleverde prestaties en functioneren voordeden. De oplossing hiervoor moest toepasbaar zijn over de hele lijn van producten en diensten waarover het bedrijf beschikte. De software zelf moest op zijn beurt verstaanbaar zijn voor zowel de ontwikkelaars, ontwerpers evenals collega’s die niet over technisch voorkennis beschikten. Na enkele jaren intern te hanteren is deze software publiek gesteld in 2015. Facebook hun doel van de software was een makkelijkere manier ontwikkelen om benodigde data op te halen zonder dat hun applicatie ontwikkelaars weten welke bronnen er exact gehanteerd werden.\autocite{GraphQLFoundation2022}

\subsection{\IfLanguageName{dutch}{Querytaal}{Query language}}%
\label{sec:Querytaal}
De naam GraphQL is een samenstelling van twee woorden, Graph en QL. Het deel QL staat voor Query Language en zal benoemd worden doorheen deze bachelorsproef als querytaal. Een querytaal is een manier van schrijven met als doel data of informatie uit één of meerdere tabellen van een databank te verkrijgen. In de meeste instanties bestaat een databank uit verschillende rijen en tabellen bevattende data omtrent een speciefiek thema zoals gegevens van werknemers. De data verkrijgen verloopt aan de hand van een verzoek, de query, die specificeert welke data uit de gebruikte databank benodigd is. Die query bestaat uit een vastgelegde werkwijze van coderen zodat de databank het verzoek begrijpt en kan verwerken.Via een query kan je de tabel met gegevens aanspreken om de gewenste data op te vragen, aan te passen, sorteren en uiteindelijk weergeven aan de gebruiker naar gelang de gebruikte commandos in het verzoek.

Een query binnen GraphQL is een string die verstuurd wordt naar een server. Na het ontvangen van de query zal de server deze interpreteren en het verzoek vervullen. Als de gewenste data of operatie uitgevoerd is, zal de server een JSON sturen naar de client. De queries zijn gevormd naar de structuur van het verwachte antwoord. Op die manier kan men de query opstellen naar gelang de benodigde data voor een applicatie. Een query binnen GraphQL is ook onderverdeeld in niveaus, waarbij elk niveau overeenstemt met een type dat een set van velden bevat. Deze types kunnen via een query uit een GraphQL server opgevraagd worden. Door dat de query personaliseerdbaar is qua vorm naar de opstelling van de client, kan men de servers simplistischer en meer gegeneraliseerd maken.\autocite{Byron2015}

\subsection{\IfLanguageName{dutch}{Graaf}{Graph}}%
\label{sec:graaf}
Het andere deel van GraphQL's naam is Graph of graaf. Een graaf wordt gedefinieerd als een object bestaande uit een verzameling van een groep punten benoemd als knopen en hun onderlinge verbindingen, genaamd bogen. Elke boog zorgt voor de verbinding van twee knopen. Een verbinding kan ook een pijl bevatten om zijn richting aan te tonen. Op deze manier kunnen de relaties tussen verscheidene knopen in beeld gebracht worden. De boog op zich kan ook nog extra informatie bevatten, in dat geval hebben we een gewogen graaf. Binnen omgevingen waar men data ordent volgens een hiërarchische structuur zoals bestandssystemen, maakt men gebruik van bomen en grafen om de data te modelleren.\autocite{Lievens2021} In het werkstuk van \textcite{Brysbaert2021} wordt dit voorgesteld als een schema waarbij de knopen gebruikers zijn van het sociale media platform facebook en bogen de onderlinge vriendschap voorstellen. De bogen bevatten ook een pijl naar beide richtingen, omdat een vriendschap op het platform een wederzijdse toepassing is. In dit geval is de graaf dus ongericht. Dit is niet altijd een vereiste.

De hierboven uitgelegde werking kan men ook toepassen op een bestaande databank. Er zijn verbindingen tussen verschillende soorten data die elk ook hun specieke kenmerken bevatten. Via GraphQL kan men dus hierop inspelen en dichter op de actuele werking van een databank te werk gaan.
Juist omdat grafen zo dicht aansluiten bij de realiteit als men een model opstelt in gedachte houdende de processen die men moet doorlopen, kan men dit implementeren in een werkomgeving. Gebruikmakende van GraphQL, kan men een model opstellen volgens een graaf steunend op een schema. Binnen dit schema worden de knopen en hun onderlinge relaties vastgelegd. Aan de client zijde kan dit dan een vergelijkende weergave vormen zoals bij object georiënteerd programmeren. Enerzijds types die onderlinge referenties bevatten en anderzijds kan men voor de backend zowel hun oude of nieuwe instanties gebruiken.\autocite{GraphQLFoundation2022}

\subsection{\IfLanguageName{dutch}{Gebruik}{Usage}}%
\label{sec:Gebruik}
Bij de sterke punten van GraphQL hoort toch wel de preventie van over- en underfetching. Overfetching is een probleem die zich vaak voordoet bij eerder traditionele programmas zoals REST, waarbij er te veel data opgevraagd wordt ten opzichte van wat benodigd is. Gebruikmakend van GraphQL kan men zich toespitsen op juist de data die van toepassing is. Dit zorgt voor een lager verbuik van bandbreedte, dat voordelig is als men ook gebruikt maakt van mobiele aparaten zoals ook Facebook doet. Het omgekeerde kan zich echter ook voor doen, bij underfetching wordt niet alle opgevraagde data weergeven in één keer. Dit kan voorgesteld worden als een boodschappenlijstje bij een grootwarenhuis of online winkel waarbij er voor elke productbeschrijving een individuele query zou moeten uitgevoerd worden. Dit is ook een probleem waar GraphQL op inspeeld door gebruik te maken van een hiërarchische opstelling. De onderlinge relaties tussen objecten wordt zo natuurlijk mogelijk behouden. \autocite{Byron2015}

Voor ontwikkelaars is het ook handig dat GraphQL declaratief is, op deze manier is de data makkelijker te hanteren en zijn de queries overzichtelijker. Men kan ook gebruik maken van nesting om gerelateerde data op te vragen en om de queries zelf consistent te houden gedurende het hele proces. Via deze werkwijze moeten er dan ook geen responses samengevoegd worden. GraphQL bezit ook schemas die kunnen dienen als een contract om front-end apps te ontwikkelen (dit gebeurd door de API verzoeken te simuleren). Het back-end team kan het contract dan later aanleveren met de nodige diensten. Binnen een graaf moeten de ontwerpers maar over een enkele endpoint beschikken om toegang te hebben tot de achterliggende data.

\subsection{\IfLanguageName{dutch}{API}{API}}%
\label{sec:API}
Een Application Programming Interface, beter gekend als API, staat in voor de communicatie tussen een client en een server. Vanuit de client wordt er een verzoek gestuurd naar de corresponderende server met de benodigde gegevens voor het verzonden request. De server op zijn beurt zal dan het verzoek verwerken en een gepaste respons sturen naar de client. Hierna zal de respons omgevormt worden en beschikbaar gesteld worden aan de gebruiker op een duidelijke wijze. Via de vernoemde client kan men gebruik maken van een request om gegevens naar een databank te sturen of om data juist op te vragen. Dit gebeurd aan de hand van een API. Deze is als het ware de tussenpersoon tussen de client en de server. \autocite{Willem2021}
