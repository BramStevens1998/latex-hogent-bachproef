%%=============================================================================
%% Samenvatting
%%=============================================================================

% TODO: De "abstract" of samenvatting is een kernachtige (~ 1 blz. voor een
% thesis) synthese van het document.
%
% Een goede abstract biedt een kernachtig antwoord op volgende vragen:
%
% 1. Waarover gaat de bachelorproef?
% 2. Waarom heb je er over geschreven?
% 3. Hoe heb je het onderzoek uitgevoerd?
% 4. Wat waren de resultaten? Wat blijkt uit je onderzoek?
% 5. Wat betekenen je resultaten? Wat is de relevantie voor het werkveld?
%
% Daarom bestaat een abstract uit volgende componenten:
%
% - inleiding + kaderen thema
% - probleemstelling
% - (centrale) onderzoeksvraag
% - onderzoeksdoelstelling
% - methodologie
% - resultaten (beperk tot de belangrijkste, relevant voor de onderzoeksvraag)
% - conclusies, aanbevelingen, beperkingen
%
% LET OP! Een samenvatting is GEEN voorwoord!

%%---------- Nederlandse samenvatting -----------------------------------------
%
% TODO: Als je je bachelorproef in het Engels schrijft, moet je eerst een
% Nederlandse samenvatting invoegen. Haal daarvoor onderstaande code uit
% commentaar.
% Wie zijn bachelorproef in het Nederlands schrijft, kan dit negeren, de inhoud
% wordt niet in het document ingevoegd.

\IfLanguageName{english}{%
\selectlanguage{dutch}
\chapter*{Samenvatting}
Deze bachelorproef kan gebruikt worden binnen Delaware om GraphQL functionaliteit toe te voegen aan de Data-Accelerator. Dit kan een goede uitbreiding voor de accelerator zijn omdat met alleen het gebruik van de Data-Accelerator er nog heel wat potentie omtrent queries en databewerking te behalen is. Het doel van deze studie is om na te gaan of GraphQL kan geïmplementeerd worden in de omgeving van de accelerator en of dit op een haalbare wijze bij projecten kan gerealiseerd worden.

Daarnaast wordt er ook bekeken of dit proces, die nu volledig handmatig verloopt, geautomatiseerd kan worden. Op die manier kan de implementatie van GraphQL bij een klant op een vlotte en betaalbare wijze verlopen. Ook voor collega's die eens willen testen hoe de accelerator in zijn werk gaat, kan er via opties bij de pipeline een omgeving opgezet worden waar deze kunnen testen hoe de Data-Accelerator gebruikt kan worden en hoe GraphQL hier een meerwaarde bij is.

In deze studie bevindt zich eerst en vooral een inleiding, deze wordt opgevolgd door de literatuurstudie of stand van zaken. Binnen de methodologie kunnen de doorlopen stappen bekeken worden evenals het resultaat van dit onderzoek.

Uit dit onderzoek blijkt dat men gemiddeld op slechts 15 minuten een omgeving kan opzetten waar GraphQL aan toegevoegd kan worden voor verder gebruik. Deze omgeving beschikt ook over alle nodige resources binnen Azure om dit op een veilge manier teweeg te brengen. Er kan wel in toekomstig onderzoek nog uitgewezen worden in hoeverre het gebruik van GraphQL tot de limiet gebracht kan worden met de Data-Accelerator en het gebruik van Logic en Function Apps.
\selectlanguage{english}
}{}

%%---------- Samenvatting -----------------------------------------------------
% De samenvatting in de hoofdtaal van het document

\chapter*{\IfLanguageName{dutch}{Samenvatting}{Abstract}}


